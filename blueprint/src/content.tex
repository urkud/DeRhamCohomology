\chapter{Continuous alternating maps}

\begin{definition}%
  \label{def:cont-alt-map}
  \lean{ContinuousAlternatingMap}
  \leanok%
  A \emph{continuous alternating map} from \(E^{I}\) to \(F\),
  where \(E\) and \(F\) are topological vector spaces
  over a nontrivially normed field \(\mathbb k\)
  and \(I\) is an index type (usually, \texttt{Fin n} for some \(n\)),
  is a map \(f\colon E^{I} \to F\) such that
  \begin{itemize}
  \item \(f(x_{1}, \dots, x_{n - 1})\) is linear in each variable \(x_{i}\);
  \item \(f\) is a continuous map.
  \end{itemize}
\end{definition}

We will use notation\todo{Better \LaTeX\ notation.} \(E [\bigwedge^{I}]\to L[\mathbb k] F\)
for the type of continuous alternating maps from \(E^{I}\) to \(F\).
We will also write (in the blueprint but not in Lean code)
\(E [\bigwedge^{k}]\to L[\mathbb k] F\) for \(E [\bigwedge^{\Fin k}]\to L[\mathbb k] F\).

\section{Pointwise operations}%
\label{sec:pointwise-operations}

\subsection{Arithmetic}%
\label{sec:arithmetic}

\begin{theorem}%
  \label{thm:normed-grp}
  If \(E\) is a seminormed space over \(\mathbb k\) and \(F\) is a (semi)normed space over \(\mathbb k\),
  then \(k\)-alternating maps from \(E\) to \(F\) form a (semi)normed space over \(\mathbb k\).
  The addition and scalar multiplication are pointwise, and the norm is given by
  \begin{equation}
    \label{eq:alt-norm}
    \|f\| = \operatorname{Inf} \left\{C \mid C \ge 0 \wedge \forall m, \|f(m)\| \le C\prod_{i=0}^{k-1}\|m_{i}\|\right\}.
  \end{equation}
\end{theorem}

\subsection{Prod, pi}%
\label{sec:prod-pi}

\todo[inline]{Define prod, pi (what else?) of CAMs}

\subsection{Composition}%
\label{sec:composition}

\begin{theorem}%
  \label{thm:clm-comp}
  If \(g\colon F \to G\) is a continuous linear map between topological vector spaces over \(\mathbb k\)
  and \(f\colon E [\bigwedge^{I}]\to L[\mathbb k] F\) is a continuous altenrating map,
  then \(g \circ f\colon E [\bigwedge^{I}]\to L[\mathbb k] G\) is a continuous alternating map.
\end{theorem}

\begin{theorem}%
  \label{thm:comp-clm}
  If \(g\colon F [\bigwedge^{I}]\to L[\mathbb k] G\) is a continuous altenrating map
  and \(f\colon E \to F\) is a continuous linear map
  then \(f^{*}g \colon E [\bigwedge^{I}]\to L[\mathbb k] G\) given by \(f^{*}g(\dots, x_{j}, \dots) = g (\dots, f(x_{j}), \dots)\) is a continuous alternating map.
\end{theorem}

\subsection{Flipping the arguments}%
\label{sec:flipping-arguments}

\todo[inline]{Flip CLM with CAM, both ways}
\todo[inline]{Flip CAM with CAM}

\subsection{Low order}%
\label{sec:low-order}

For \(k = 0\) and \(k = 1\), there are simple descriptions of the space of continuous altenrating forms,
given by the following definitions.

\begin{definition}
  If \(I\) is an empty type (e.g., \(\Fin 0\)),
  then the map sending the unique element of \(E^{I}\) to a constant \(c\)
  is a continuous alternating map.
  Moreover, this construction provides a continuous linear equivalence \(\mkZero\)
  between \(F\) and \(E [\bigwedge^{I}]\to L[\mathbb k] G\).
\end{definition}

\begin{definition}
  If \(I\) has a unique element \(i\) (e.g., \(I = \Fin 1\) and \(i = 0\)),
  then continuous alternating maps from \(E^{I}\) to \(F\) are just continuous linear maps from \(E\) to \(F\).
  The equivalence is given by \(\mkOne\).
\end{definition}

\subsection{Finite dimensional}%
\label{sec:finite-dimensional}

\begin{theorem}
  If the domain \(E\) is a finite dimensional topological vector space with Hausdorff topology,
  then any multilinear map from \(E\) is continuous.
\end{theorem}

\begin{theorem}
  If the domain \(E\) is a finite dimensional topological vector space with Hausdorff topology,
  then any alternating map from \(E\) is continuous.
\end{theorem}

\subsection{Basis}%
\label{sec:basis}

\todo[inline]{Define a basis on the space of (continuous) alternating maps given bases in the domain and in the codomain.}

\subsection{Alternatization}%
\label{sec:alternatization}

\todo[inline]{Write a section about alternatization and why we rarely use it (avoid division by currying instead).}

\subsection{Currying}%
\label{sec:currying}

\begin{theorem}
  If \(f\colon E [\bigwedge^{k + 1}]\to L[\mathbb k] F\) is a \((k + 1)\)-alternating map and \(v\) is a vector,
  then \(f_{v}\colon E [\bigwedge^{k}]\to L[\mathbb k] F\) given by
  \[
    f_{v}(x_{0}, \dots, x_{k - 1})=f(v, x_{0}, \dots, x_{k - 1})
  \]
  is a \(k\)-alternating map.
  Moreover, this map is linear in \(f\) and \(v\), so we can define
  \begin{align}
    \curryOne&\colon (E [\bigwedge^{k + 1}]\to L[\mathbb k] F) \to L[\mathbb k] E \to L[\mathbb k] E [\bigwedge^{k}]\to L[\mathbb k] F\\
    (\curryOne f)(v)(x) &= f(v, x).
  \end{align}
\end{theorem}

\todo[inline]{Define currying for \(k + l\)-alternating maps.}
\begin{definition}%
  \label{def:alt-one}
  Let \(f\colon E^{k + 1}\to F\) be a continuous multilinear map which is alternating in all but the first variable.
  Formally, we represent this as \(f\) being a continuous linear map on \(E\) taking values in the space of continuous alternating maps \(E^{k}\to F\).
  Define the \emph{uncurrying} or \emph{\(1\)-alternatization} of \(f\) to be the continuous alternating map given by
  \[
    \altOne(f)(x_{0}, \dots, x_{k}) = \sum_{j=0}^{k}{(-1)}^{j}f(x_{j}, x_{0}, \dots, \widehat{x_{j}}, \dots, x_{k}),
  \]
  where \(\widehat{x_{j}}\) means that \(x_{j}\) is removed from the sequence.
  Here \({(-1)}^{j}\) is the sign of the cyclic permutation \((0, \dots, j)\).
\end{definition}

\begin{theorem}
  If \(f\) is a continuous \(k\)-alternating map, then \(\altOne(\curryOne f) = (k + 1)f\).
\end{theorem}

\begin{remark}
  We do not divide the result of \autoref{def:alt-one} by \(k + 1\)
  to ensure that it works if \(\mathbb k\) has a positive characteristic.%
  \todo{Are there any nontrivially normed fields of positive characteristic?}
\end{remark}

\begin{theorem}
  Let \(f\colon E^{k + 2}\to F\) be a continuous multilinear map which is alternating in all but the first two variables.
  If \(f\) is symmetric in the first two variables, then \(\altOne(\altOne(f)) = 0\).
\end{theorem}


\begin{definition}
  Let \(f\colon E^{k + l}\to F\) be a continuous multilinear map
  which is alternating in the first \(k\) argument and in the other \(l\) arguments.
  Then \(\altAdd_{k,l}(f)\) is the continuous alternating map given by
  \[
    \altAdd_{k,l}(f)(x_{0}, \dots, x_{k + l - 1})=\sum_{\substack{s\subseteq [0, k + l)\\|s|=k}}{(-1)}^{\sign \sigma_{s}}f(\sigma_{s}(0), \dots, \sigma_{s}(k + l - 1)),
  \]
  where \(\sigma_{s}\) is a permutation of \(\Fin (k + l)\) that sends \(s\) to \([0, k)\) and its complement to \([k, k + l)\).
  The exact choice of \(\sigma_{s}\) is not important, but we choose the permutation that is monotone both on \(s\) and on its complement.
\end{definition}

\begin{theorem}
  If \(k = 1\), then \(\altAdd_{1, l}(\mkOne(f)) = \altAdd_{1}(f)\).
\end{theorem}

\begin{theorem}
  If \(f\) is a continuous alternating map, then \(\curry_{k, l}(\altAdd_{k, l}(f)) = \binom{k + l}{k}f\).
\end{theorem}

\section{Wedge product}%
\label{sec:wedge-product}

\begin{definition}
  Consider a continuous bilinear map \(f \colon E' \times F' \to G\)
  and continuous alternating maps \(g\colon E^{I}\to E'\) and \(h\colon F^{J}\to F'\).
  Then the formula
  \[
    f \circ_{2} (g, h)(v, w) = f(g(v), h(w))
  \]
  defines a continuous multilinear map \(E^{I}\times F^{J}\to G\)
  which is alternating both in the first \(I\) variables and in the other \(J\) variables.
\end{definition}

\begin{definition}
  Let \(f \colon F \times G \to H\) be a continuous bilinear map.
  Let \(\omega \colon E^{k} \to F\) and \(\eta\colon E^{l}\to G\) be continuous alternating maps.
  Then the \emph{wedge product} of \(\omega\) and \(\eta\) with respect to \(f\) is given by
  \[
    \omega \wedge_{f} \eta = \altAdd_{k, l}(f \circ_{2}(g, h)).
  \]
\end{definition}

The most common case is \(F = G = H = \mathbb k\) and \(f\) is the multiplication,
but other bilinear maps (e.g., the inner product on a Hilbert space) are useful in some cases.
We use notation \(\omega \wedge \eta\) for \(f(a, b) = ab\).

\todo[inline]{Add obvious theorems about dependency of \(\omega \wedge_{f} \eta\)on \(f\). E.g., \(\omega \wedge_{g \circ f} \eta = g \circ (\omega \wedge_{f} \eta)\).}

\begin{theorem}
  Wedge product \(\omega \wedge_{f} \eta\) is linear in \(f\), \(\omega\), and \(\eta\).
  \todo{Do we need 3 theorems to link to 3 Lean theorems? Or should we create a huge bundled continuous linear map?}
\end{theorem}

\begin{theorem}
  The wedge product is associative\todo{Add a version for \(\omega \wedge_{f}\eta\); we'll need several spaces and continuous bilinear maps here}, \((\omega_{1}\wedge \omega_{2})\wedge \omega_{3} = \omega_{1}\wedge (\omega_{2}\wedge \omega_{3})\).
  Note that in Lean the sides have different types as \((k + l) + m = k + (l + m)\) is not a definitional equality,
  so the formalized statement involves a natural isomorphism.
\end{theorem}

\begin{theorem}
  If \(a\) is a vector and \(\omega\) is a continuous alternating map,
  then \(\mkZero a \wedge_{f} \omega = f(a) \circ \omega\).
\end{theorem}

\begin{theorem}
  If \(a\) is a number and \(\omega\) is a continuous alternating map,
  then \(\mkZero a \wedge_{f} \omega = f(a) \circ \omega\).
\end{theorem}

\begin{theorem}
  If \(\omega\colon E^{k}\to F\) and \(\eta\colon E^{l}\to G\) are continuous alternating maps
  and \(f \colon F \times G \to H\) is a continuous bilinear map,
  then \(\omega \wedge_{f} \eta = {(-1)}^{kl}\eta \wedge_{f'} \omega\),
  where \(f'(y, x) = f(x, y)\).
  Note that the LHS and the RHS have different types in Lean,
  so the actual statement involves a natural isomorphism.
\end{theorem}

We will need the following definition, if we want to define contact forms and symplectic forms:

\begin{theorem}
  Let \(\omega\colon E^{2}\to \mathbb R\) be a skew-symmetric bilinear map on a vector space of dimension \(2k\).
  Then \(\omega\wedge\dots\wedge\omega = 0\) iff the corresponding matrix has determinant zero.
  \todo{We may need \(n\)-ary version of the wedge product to define this. Or should we just define the power by induction?}
\end{theorem}

\chapter{Differential forms on a normed space}%
\label{cha:differential-forms-normed-space}

In this chapter we study differential forms on a normed vector space.
We use notation \(\Omega^{k}(E, F)\) for the space of differential \(k\)-forms on \(E\) taking values in \(F\),
where a \(k\)-form is a map from \(E\) to \(E [\bigwedge^{I}]\to L[\mathbb k] F\).

For now, we only consider real \(k\)-forms, though we may generalize the notion to any field in the future.
\todo{Generalize \(k\)-forms to any field; if a field isn't \(\mathbb R\) or \(\mathbb C\), then only analytic \(k\)-forms satisfy \(d^{2}=0\).}

All pointwise operations (arithmetic, compositions with continuous linear map, currying).

\section{Pullback}%
\label{sec:pullback}

\begin{definition}
  The \emph{pullback} of \(\omega\in \Omega^{k}(F, G)\) along a map \(f\colon E\to F\) within a set \(s\)
  is the \(k\)-form \(f_{s}^{*}\omega\) given by \(f^{*}\omega(x) = {(d_{s}f(x))}^{*}\omega(f(x))\),
  where \(d_{s}f(x)\) is the Fréchet derivative of \(f\) within \(s\) at \(x\),
  and the right-hand side is defined in~\autoref{thm:comp-clm}.
\end{definition}

\begin{definition}
  The \emph{pullback} of \(\omega\in \Omega^{k}(F, G)\) along a map \(f\colon E\to F\)
  is the pullback of \(\omega\) along \(f\) within the universal set in the domain.
\end{definition}

\begin{theorem}
  The pullback \(f_{s}^{*}\omega\) within a set is a continuous linear map in \(\omega\).
\end{theorem}

\begin{theorem}
  The pullback \(f^{*}\omega\) is a continuous linear map in \(\omega\).
\end{theorem}

\begin{theorem}
  The pullback (within a set) of the \(0\)-form given by a function \(f\colon F\to G\) along a map \(g\colon E\to F\)
  is the \(0\)-form given by the function \(f\circ g\).

  Note that this statement is true even if \(g\) is not differentiable.
\end{theorem}

\todo[inline]{Formulate a similar statement about the pullback of a \(1\)-form.}
\todo[inline]{Formulate a similar statement about the pullback of the wedge product of forms.}

\section{Exterior derivative}%
\label{sec:exterior-derivative}

\begin{definition}%
  \label{def:ederivWithin}
  Consider a differential \(k\)-form \(\omega \in \Omega^{k}(E, F)\).
  By definition, its Fréchet derivative within a set \(s\) at a point \(x\)
  is a continuous linear map taking values in the space of continuous alternating maps from \(E^{k}\) to \(F\).
  The \emph{exterior derivative} of \(\omega\) \emph{within} \(s\) is the \((k + 1)\)-form \(d_{s}\omega\in\Omega^{k + 1}(E, F)\)
  that sends each \(x\) to the uncurrying of this Fréchet derivative.
\end{definition}

\begin{definition}%
  \label{def:ederiv}
  The \emph{exterior derivative} of a differential \(k\)-form \(\omega\)
  is its exterior derivative within the universal set in the domain.
\end{definition}

\begin{theorem}
  Assume that \(\omega\) is twice differentiable within \(s\) at \(x\)
  and the second derivatives commute. Then \(d_{s}(d_{s}\omega) = 0\).
  The latter condition automatically holds, if the base field is
  \(\mathbb R\) or \(\mathbb C\) and \(s\) is a convex set with
  nonempty interior.
\end{theorem}

\begin{theorem}
  The pullback commutes with the exterior derivative.
  \todo[inline]{What are the exact assumptions?}
\end{theorem}

\begin{proof}
  For simplicity, we only prove this theorem
  for the non-restricted exterior derivative and pullback.

  Consider a differentiable function \(f\colon E \to F\)
  and a differentiable \(k\)-form \(\omega\in \Omega^{k}(F, G)\).
  Take \(k+1\) vectors \(X_{0}, \dots, X_{k}\in E\).
  Simplify both sides.
  \begin{align*}
    (f^{*}d\omega(a))(X_{0}, \dots, X_{k})
    &=(d\omega)(f(a))(f_{*}(a)X_{0}, \dots, f_{*}(a)X_{k})\\
    &=\sum_{j=0}^{k}{(-1)}^{j}\partial_{f_{*}(a)X_{j}}\omega(y)(\dots, \widehat{f_{*}(a)X_{j}}, \dots)
  \end{align*}
  OTOH,
  \begin{align*}
    d(f^{*}\omega)(a)(X_{0}, \dots, X_{k})
    &=\sum_{j=0}^{k}{(-1)}^{j}\partial_{X_{j}}(f^{*}\omega)(\dots, \widehat{X_{j}}, \dots)\\
    &=\sum_{j=0}^{k}{(-1)}^{j}\partial_{X_{j}}\left[\omega(f(x))(\dots, \widehat{f_{*}(x)X_{j}}, \dots)\right]\\
    &=(f^{*}d\omega(a))(X_{0}, \dots, X_{k}) + \sum_{j=0}^{k}{(-1)}^{j}\partial_{X_{j}}\left[\omega(f(a))(\dots, \widehat{f_{*}(x)X_{j}}, \dots)\right]\\
  \end{align*}
  \todo[inline]{It looks like we can prove a formula that implies both this and the formula for \(d\omega(X_{0}, \dots, X_{k})\), where \(X_{j}\) are vector fields.}
\end{proof}
\chapter{Vector bundle of continuous alternating maps}%
\label{cha:vect-bundle-cont}

\chapter{Differential forms on a manifold}%
\label{cha:diff-forms-manif}

